\section{Introduction}

It has been observed that that changing the avidity of an immune complex (IC)---that is, the number of epitopes on the surface of the complex which are able to bind to a receptor or antibody in question---presenting IgG domains to cells that express receptors of the Fc\textgamma{} class (Fc\textgamma{}Rs) strongly influences the number of ICs bound to the cell at equilibrium \cite{Lux:2013iv, Bruhns:2009kg}.  For the sake of understanding the equilibrium binding of ICs to cell surfaces, an IC is typically modeled as a multivalent ligand. While extensive and very useful work has been done to model the binding of multivalent ligand to receptors expressed on a cell membrane, these models are very short-sighted in that they only take into account ligand expressing a single kind of epitope binding to only a single kind of receptor. Furthermore, most of these models take advantage of theoretical cross-linking constants which, though numerically solvable, have yet to be expressed as functions of other known physical parameters (ligand avidity, constants of association for monovalent epitope-receptor binding, etc.).

Extensive research has been done in the way of exploring the impact of avidity in the binding of ICs to receptors on cell membranes. One publication in this branch of study demonstrates the potential of multivalent immune complexes to discriminate between receptors on membranes expressing multiple receptor types with non-negligible affinity to a particular epitope. In other words, by altering the avidity of an IC, it is possible to significantly increase its equilibrium binding to a particular receptor while decreasing its equilibrium binding to most or all other receptors of non-negligible affinity to the same epitope such that the difference in the order of magnitude of the ratio of ligand bound to the former over ligand bound to any other might be made larger than would be the case using monovalent receptor alone. Our research also demonstrates that controlling both the avidity and aforementioned cross-linking constant allows for even greater discrimination in IC binding between receptors of that bind with the IC non-negligibly.