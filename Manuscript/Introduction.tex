\section{Introduction}

% This paragraph addresses the idea that IgGs are relevant to disease
Antibodies are critical and central regulators of the immune response. Antibodies of the IgG isotype interact with FcgR receptors expressed widely on innate immune effector cells. Regulation of effector cell function is a critical component of the IgG therapy's use in cancer and autoimmune diseases. Effector function operates through multiple cell types---including macrophages, monocytes, neutrophils, and NK cells---and multiple process such as antibody-dependent cell-mediated cytotoxicity (ADCC), promoting antigen presentation, and cytokine response. In addition to their effect in isolation, IgG therapies can show synergistic effect in cancers in combination with checkpoint and cytokine-mediated immunotherapies \ac{Moynihan:2016jb, Zhu:2015gy}. These immunotherapeutic effects in combination with antibodies' ability to operate as signaling modulators through competitive binding and opsonization make IgG biologic agents particularly versatile therapeutic agents.

% This paragraph addresses the idea that efforts have been made to manipulate IgG function
An ability to quantitatively predict IgG-\fcgr{} function would aid understanding and treatment of cancer and autoimmune diseases. Efforts to engineer improved effector responses to IgG treatment include mutated Fc variants with biased \fcgr{} binding, deglycosylated Fc domains with the effect of modulating \fcgr{} binding, and alternative IgG subclasses with distinct binding profiles. In addition, particularly in cases where antigen and antibody are exogenously provided, avidity and binding affinities may be manipulated coordinately. % These have generally had X limitations through...

% This paragraph addresses the idea that people have tried to understand IgG function
Multiple efforts have sought to improve our understanding of IgG-mediated effector function. These include efforts to carefully quantify the individual, monovalent \fcgr{}-IgG affinities \ac{Bruhns:2009kg}. In addition, previous studies have characterized the effects of IgG glycosylation (which serves to modulate \fcgr{} affinity) and immune complex avidity on the binding of IgG-antigen complexes \ac{Lux:2013iv}. Genetic models have made it possible to remove certain \fcgr{}s and examine the consequent effect on IgG treatment in various cancer models\cite{Clynes:2000ga}. Finally, comparison of antibodies with matched variable regions but differing Fc domains has allowed the effects to be compared, through with necessarily pleiotropic differences between each of these classes \ac{Nimmerjahn:2005hu}.

% This paragraph addresses the idea that quantitative models have been used to understand immune receptors
Models of multivalent ligand binding to monovalent receptors have been successfully employed to study function of other immune receptors with known, corresponding binding models\ac{Perelson:1980fs, Perelson:1980ds, Hlavacek:1999gb}. For example, a two-component binding model can capture the effect of T cell receptor activation \ac{Stone:2001fm}. However, the \fcgr{} family presents the considerable additional challenge of multiple distinct receptor classes being expressed simultaneously within cells. Additionally, the multiple \fcgr{}s present, with activating and inhibitory roles, ensure that any manipulation of immune complex composition will necessarily have multivariate effects. The same challenge exists for other paired receptor-ligand families, including other immunoglobulin classes and the many phosphatidylserine receptors. Thus, while the underlying theoretical models of multivalent binding are long-standing, \fcgr{}-IgG interactions are especially suited for developments in the ability to rigorously link these models to experimental observations.

In this study, we have employed a model of multivalent immune complex binding to FcgR receptors and show that it can capture the experimentally measured binding at differing valencies. Applying this model, we show it can quantitatively predict effector response in response to diverse interventions in a forward manner, and can deconvolve the causal factors of response in a reverse fashion. More broadly, these results demonstrate the power of a unified binding model tied to computational inference techniques linking theory and experimental observation.
